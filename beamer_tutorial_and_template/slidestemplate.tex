\documentclass[13pt, notes, aspectratio=169]{beamer}

\usepackage{pgfpages}
%\pgfpagesuselayout{2 on 1}[a4paper,border shrink=5mm]
	% to amake a handout you can pout "handout" in the doc class, but things get kind of weird with the overlays  
%\setbeameroption{show notes on second screen=right} % Both
\setbeameroption{hide notes}

\usepackage{verbatim}

\usepackage[T1]{fontenc}
\usepackage{array}
\usepackage{amsmath}
\usepackage{mathpazo}
\usepackage{hyperref}
\usepackage{lipsum}
\usepackage{multimedia}
\usepackage{graphicx}
\usepackage{multirow}
\usepackage{graphicx}
\usepackage{dcolumn}
\usepackage{booktabs}
\usepackage{bbm}
\usepackage{pifont}
\usepackage{tikz}\usetikzlibrary{tikzmark,calc,,arrows,shapes,decorations.pathreplacing}
\tikzset{every picture/.style={remember picture}}

\usepackage{changepage}
\usepackage{appendixnumberbeamer}
\newcommand{\beginbackup}{
   \newcounter{framenumbervorappendix}
   \setcounter{framenumbervorappendix}{\value{framenumber}}
   \setbeamertemplate{footline}
   {
     \leavevmode%
     \hline
     box{%
       \begin{beamercolorbox}[wd=\paperwidth,ht=2.25ex,dp=1ex,right]{footlinecolor}%
%         \insertframenumber  \hspace*{2ex} 
       \end{beamercolorbox}}%
     \vskip0pt%
   }
 }
\newcommand{\backupend}{
   \addtocounter{framenumbervorappendix}{-\value{framenumber}}
   \addtocounter{framenumber}{\value{framenumbervorappendix}} 
}


\usepackage{graphicx}
\usepackage[space]{grffile}

\definecolor{blue}{RGB}{3,174,240}
%\definecolor{red}{RGB}{213,94,0}
\definecolor{red}{RGB}{204,121,167}
\definecolor{yellow}{RGB}{240,228,66}
\definecolor{green}{RGB}{0,158,115}

\hypersetup{
  colorlinks=false,
  linkbordercolor = {white},
  linkcolor = {red}
}


%% I use a beige off white for my background
\definecolor{MyBackground}{RGB}{255,253,218}
\definecolor{lightperiwinkle}{RGB}{194,194, 219} %light periwinkle
\definecolor{middlebluepurple}{RGB}{126, 126, 180} %middle blue purple
\definecolor{paradisepink}{RGB}{224	, 82, 99} %paradisepink
\definecolor{amaranth}{RGB}{221, 64, 83}


%% Uncomment this if you want to change the background color to something else
% \setbeamercolor{background canvas}{bg=MyBackground}

%% Change the bg color to adjust your transition slide background color!
\newenvironment{transitionframe}{
  \setbeamercolor{background canvas}{bg=lightperiwinkle}
  \color{black} 
  \begin{frame} 
  		\huge \fontfamily{qag}\selectfont  \centering }{
    \end{frame}
}

\usepackage{tgheros}

% \setbeamercolor{frametitle}{fg=black}
\setbeamerfont{title}{series = \bfseries\itshape, family = {\fontfamily{qag}}}
\setbeamerfont{frametitle}{series = \itshape\bfseries, family = {\fontfamily{qag}}}
%\setbeamercolor{title}{fg=black}
\setbeamertemplate{footline}[frame number]
\setbeamertemplate{navigation symbols}{} 

% MY CUTE ITEMIZE STARS AND DIAMONDS:
\setbeamertemplate{itemize item}{\small \ding{80}}
\setbeamertemplate{itemize subitem}{\scriptsize \ding{70}}
	% maybe you want something more standard:
	 % \setbeamertemplate{itemize item}{\blacktriangleright}
	 % \setbeamertemplate{itemize subitem}{\triangleright}
	 		% helpful for lists: https://www.overleaf.com/learn/latex/Lists

\setbeamercolor{itemize item}{fg = black}
\setbeamercolor{enumerate item}{fg = black} 
\setbeamercolor{itemize subitem}{fg = black}
%\setbeamercolor{itemize subitem}{fg=black}
%\setbeamercolor{enumerate item}{fg=black}
%\setbeamercolor{enumerate subitem}{fg=black}
%\setbeamercolor{button}{bg=blue,fg=white,}
\usecolortheme{seahorse}


\setbeamersize{text margin left=2em,text margin right=2em} 

\newenvironment{wideitemize}{\itemize\addtolength{\itemsep}{10pt}}{\enditemize}

\title{\huge a mini tutorial + my custom template}
\author{\ding{96} emily case \ding{96}}
\date{}
\begin{document}

\begin{frame}
	\maketitle
\end{frame}

\begin{transitionframe}
	\textit{\textbf{huge shout out to Paul Goldsmith-Pinkham, who's beamer slide template provided a wonderful way to learn, as well as the foundation for this template. }}
\end{transitionframe}

\begin{frame} 
	\centering 
	\textbf{\Huge \color{amaranth}did you know that a young child rejects economics as a potential career path every time you don't use the 169 aspect ratio?} \\
	{\scriptsize (just a fun fact that is totally true.) }
\end{frame}

\begin{frame}{cuter itemized lists ----------------------------------------------------------------}
\begin{wideitemize}
	\item I use Paul's \texttt{wideitemize} environment for better spacing
	\item I like to use a lot of \textbf{bold words} and \textbf{\color{amaranth}bold, colorful words} in order to emphasize. 
	\begin{itemize}
		\item you can define your own colors, I called mine \textcolor{amaranth}{amaranth}
	\end{itemize} 
	\item you can, of course, change the stars and diamonds in the preamble. 
	\item[] occasionally I use itemized lists without any kind of symbol, to make use of the spacing. 
\end{wideitemize}
\end{frame}


\begin{frame}[fragile]{two-column slide -----------------------------------------------------------------}
\begin{columns}[T] % align columns
\begin{column}{.45\textwidth}
  woohooooooooo look at all the stuff i have going on over here!!!!!!!!!! \\
  \begin{wideitemize}
  	\item so 
  	\item much
  	\item contentttttt
  \end{wideitemize}
\end{column}%
\hfill%
\begin{column}{.5\textwidth}
	WOW it's \emph{\textbf{MORE CONTENT} OVER HERE}
		\begin{wideitemize}
			\item[\large \ding{202}] p.s. did you know you can change each individual item symbol? this one is a cute little number 1 \pause
			\item[\large \ding{203}] boom, another one \pause
			\item p.p.s., did you know about \texttt{\textbackslash pause} to iteratively display items? 
		\end{wideitemize} 		
\end{column}%
\end{columns}
\end{frame}


\begin{transitionframe}
	\textbf{\textit{sometimes i want to make an emphasized point.}}
\end{transitionframe}

\begin{frame}{going above and beyond when typing equations--------------------}
\only<1-2>{%
	it takes a long time to format your slides to do what I am about to show you, but I think it's worth it (makes things more readable + you look cool as @\#\%!) \\
	\pause
	\bigskip 
	let's look at a typical equation slide, first...
}% 
\only<3>{%
	my regression equation is
	\[ y = \beta_0 + \beta_1 x_1 + \beta_2 x_2 + \varepsilon\] 
	where 
	\begin{itemize}
		\item $y$ is the outcome variable 
		\item $\beta_0$ is the constant 
		\item $x_1$ is a variable of interest 
		\item $x_2$ is, you guessed it, another variable 
		\item \dots and so on. 
	\end{itemize}
}%
\only<4>{%
	\textbf{lets see what happens when we put some heart and soul into slide creation...}
}%
\only<5-10>{%
	my regression equation is
	\[ \textcolor<6>{amaranth}{y} = \textcolor<7>{amaranth}{\beta_0} + \beta_1 \textcolor<8>{amaranth}{x_1} + \beta_2 \textcolor<9>{amaranth}{x_2} + \varepsilon\] 
	where \pause 
	\begin{itemize}
		\item<6-> $\color<6>{amaranth} y$ is the outcome variable 
		\item<7-> $\color<7>{amaranth} \beta_0$ is the constant 
		\item<8-> $\color<8>{amaranth} x_1$ is a variable of interest 
		\item<9-> $\color<9>{amaranth} x_2$ is, you guessed it, another variable 
		\item<10> \dots and so on. 
	\end{itemize}
}%
\only<11>{%
	BONUS!!!! sometimes it's helpful to use underbrackets to highlight a section of your equation 
	\[ y = \beta_0 + \underbrace{\beta_1 x_1 + \beta_2 x_2}_{\text{this part has variables}} + \varepsilon \]
}%
\end{frame}

\begin{frame}{some notes about what just happened}
	\begin{enumerate}
		\item sometimes, I use \texttt{\color{amaranth} \textbackslash only<slidenumbers>\{slide content\}} to have \textbf{multiple slides} in the \textbf{\textit{same} frame}
		\only<1>{%
		\begin{wideitemize}
			\item \textbf{pros:} avoid re-typing your frame title, tex document might look more organized. 
			\item \textbf{cons:} have to keep track of slide numbers, which is harder to do when using \texttt{\textbackslash pause} or overlays; also does not change the slide numbers in the bottom right corner. 
		\end{wideitemize}	}%
		\only<1-2>{%
		\item I use \texttt{\textbackslash textcolor<slidenumber>\{mycolorname\}\{texttodisplay\}} to only apply a text color on certain slide numbers. 
		\begin{itemize}
			\item when the text I want to highlight is trapped within an environment, such as the math-mode dollar signs \$, instead I can use \texttt{\textbackslash color<slidenumber>\{mycolorname\}}
		\end{itemize}
		\item finally, instead of using \texttt{\textbackslash pause} I usually will tell itemize directly when to display that item: \texttt{\textbackslash item<6->} means I want this item to be displayed starting on slide 6, and continue being displayed.
		}% 
	\end{enumerate}
\end{frame}
	
\end{document}